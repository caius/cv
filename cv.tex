% LaTeX Curriculum Vitae Template
%
% Copyright (C) 2004-2009 Jason Blevins <jrblevin@sdf.lonestar.org>
% http://jblevins.org/projects/cv-template/
%
% You may use use this document as a template to create your own CV
% and you may redistribute the source code freely. No attribution is
% required in any resulting documents. I do ask that you please leave
% this notice and the above URL in the source code if you choose to
% redistribute this file.

\documentclass[letterpaper]{article}

\usepackage{hyperref}
\usepackage{geometry}

% Comment the following lines to use the default Computer Modern font
% instead of the Palatino font provided by the mathpazo package.
% Remove the 'osf' bit if you don't like the old style figures.
\usepackage[T1]{fontenc}
\usepackage[sc,osf]{mathpazo}

% Set your name here
\def\name{Caius Durling}

\def\footerlink{http://caius.name/}

% The following metadata will show up in the PDF properties
\hypersetup{
  colorlinks = true,
  urlcolor = black,
  pdfauthor = {\name},
  pdfkeywords = {},
  pdftitle = {\name: Curriculum Vitae},
  pdfsubject = {Curriculum Vitae},
  pdfpagemode = UseNone
}

\geometry{
  body={6.5in, 8.5in},
  left=1.0in,
  top=1.25in
}

% Customize page headers
\pagestyle{empty}
% \rfoot{Last updated: }

% Custom section fonts
\usepackage{sectsty}
\sectionfont{\rmfamily\mdseries\Large}
\subsectionfont{\rmfamily\mdseries\itshape\large}

% Other possible font commands include:
% \ttfamily for teletype,
% \sffamily for sans serif,
% \bfseries for bold,
% \scshape for small caps,
% \normalsize, \large, \Large, \LARGE sizes.

% Don't indent paragraphs. Space them out though.
\usepackage{parskip}

% Make lists without bullets
\renewenvironment{itemize}{
  \begin{list}{}{
    \setlength{\leftmargin}{1.5em}
  }
}{
  \end{list}
}

\begin{document}

% Place name at left
\rightline{\huge \name}
\rightline{\href{http://caius.name/}{\tt http://caius.name/}}

% Alternatively, print name centered and bold:
% \centerline{\huge \bf \name}

\section*{Blurb}

Caius is a talented problem solver who is adept at building web applications to solve difficult issues, whilst keeping them maintainable and scalable at the same time. He has been programming since his early childhood in a variety of languages and on various platforms. In the last few years he's specialised in Ruby applications, and written quite a few.

For fun outside of work he tinkers with his own projects, and enters hackdays (programming contests to build something in 24 hours, demo it to everyone else, and win prizes), with his teams winning awards or being notably mentioned at all so far.

Away from the computer he can either be found enjoying the english countryside in his sports car, taking photographs of interesting things, sampling the real ales on offer wherever he is, or simply relaxing by dinghy sailing.

\section*{Notable Projects}

\subsection*{I am, I do [2011] \href{http://iamido.info/}{\tt (http://iamido.info)}} % (fold)
\label{sub:i_am_i_do}

I am, I do was built at a weekend hackday to be an inspirational collection of users answering six poignant questions. Caius and Dom are inspired by other people and what they had achieved, or what drove them to do what they do. The app was built so that users can browse through the profiles reading other people's stories and being inspired by them, before answering those same questions on their own profile to help inspire others.

% subsection i_am_i_do (end)

\subsection*{Tweet Savr [2011] \href{http://tweetsavr.com/}{\tt (http://tweetsavr.com/)}} % (fold)
\label{sub:tweet_savr}

Tweet Savr was built in an evening and allows users to display a conversation of tweets in chronological order on one page. It also provides a permalink for that conversation to be viewed at forevermore, which the user can then share with an audience. It's a perfect example of a small tool built to solve a problem, and not expanding to do more than it needs to beyond that.

% subsection tweet_savr (end)

\subsection*{Habari [2007-2011] \href{http://habariproject.org/}{\tt (http://habariproject.org/)}} % (fold)
\label{sub:habari}

Habari aims to be the personal content management system of the future, built on state of the art technology and available to all under a free license. Caius joined the project quite early on and helped shape it towards the platform it is today. He was a founding member of the project's Cabal until he stepped back from the project in early 2011 to concentrate on other things.

% subsection habari (end)

\subsection*{/brb [2004-\the\year] \href{http://r.caius.name/brb}{\tt (http://r.caius.name/brb)}} % (fold)
\label{sub:brb}

{\tt /brb} is a very popular plugin for the leading Mac OS X instant messaging client, Adium. It inserts a humorous reason for the user to ``be right back'', to both amuse and inform the other party that the sender will be incommunicado for a short time.

% subsection  (end)

\section*{Employment}

\subsection*{PizzaPowered [2010-\the\year] \href{http://pizzapowered.com/}{\tt (http://pizzapowered.com/)}} % (fold)
\label{sub:pizzapowered}

Caius maintains the servers for the entirety of the company and it's products, as well as championing development on one of the three products. As part of his work for PizzaPowered he designed and built a resilient, well behaved web crawler for analysing people's websites without causing them undue load.

% subsection pizzapowered (end)

\subsection*{Brightbox [2008-\the\year] \href{http://brightbox.com/}{\tt (http://brightbox.com/)}} % (fold)
\label{sub:brightbox}

% subsection brightbox (end)
Caius initially helped maintain and extend the existing Ruby on Rails hosting platform, before being part of the team that designed and built the new Infrastructure-as-a-service platform which launched in October 2011. The team actively incorporated new development ideas and workflows where they enhanced the ability to solve problems, and helped share solutions and libraries they wrote for others to use.

\bigskip

% Footer
\begin{center}
  \begin{footnotesize}
    Last updated: \today
  \end{footnotesize}
\end{center}

\end{document}
